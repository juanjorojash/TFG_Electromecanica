\chapter{Conclusiones y Recomendaciones}

\section{Conclusiones}
\begin{enumerate}
    \item Se determinó que el radio transmisor representa un elemento crítico a nivel térmico dentro de la estación remota.
    \item Se construyó un modelo que describe el comportamiento  térmico del radio transmisor, según parámetros de envió.
    \item Se determinó que los parámetros máximos operativos para el módulo de potencia eran de \SI{100}{\celsius} y de \SI{60}{\celsius} para el radio transmisor.
    \item Se estableció un modelo térmico inicial de la estación remota que permite evaluar su comportamiento térmico bajo distintos parámetros operativos y ambientales.
    \item Se determinó que una primera solución térmica para el radio y la estación, es la correcta gestión del envió de datos por parte del radio transmisor.
    \item Se realizó el diseño de una solución térmica pasiva, a través de un disipador o sumidero de calor para el radio transmisor que mantiene al radio dentro del rango de operación térmica.
    \item Se posicionó estructuralmente tanto el radio como el disipador, de manera tal de que exista un interfaz mecánica entre el radio, el disipador y el gabinete.
    \item Se determinaron las acciones a seguir respecto a la humedad presente en la GRT en aras de la protección de los dispositivos electrónicos.
\end{enumerate}

\newpage

\section{Recomendaciones}

\begin{enumerate}
    \item Se debe considerar el consumo energético del radio, así como el voltaje de alimentación del mismo, de manera tal que la fuente de energía soporte los altos picos de corriente que el radio consume.
    \item El comportamiento térmico del radio mejoraría significativamente, si se realiza y se considera una estrategia enfocada en la manera de enviar los datos hacia el satélite.
    \item Una vez se tengan definidos los parámetros de envio definitivos, así como los tiempos de transmisión, se recomienda realizar un muestreo térmico de la condición final e introducir los parámetros en los distintos modelos que permitan realizar los ajustes pertinentes en caso de ser necesario.
    \item Se recomienda establecer los datos del experimento 1 como parámetros máximos no admitidos de transmisión para el radio, ya que bajo condiciones normales, representan un posible daño para la integridad del módulo de potencia.
    \item Es de suma importancia, tomar acciones respecto a la radiación solar sobre el gabinete. Una reducción de este parámetro sobre su superficie, significaría una mejoría notable en la condición térmica general de la estación y por ende del radio transmisor.
    \item Se recomienda utilizar durante el ensamblaje del disipador pasta térmica en la interfaz entre  las placas de los módulos y las barras conductoras; así como en las roscas tanto para los disipadores como para las roscas de las mismas barras, esto con el fin de evitar resistencias térmicas por contacto debido a uniones imperfectas.
    \item En el caso de las bolsas de silica gel, se recomienda cambiarlas cada vez que se realice una apertura del gabinete por motivos de mantenimiento.
    \item Es de vital importancia realizar una correcta puesta a tierra, tanto del conector de la antena, como todos aquellos elementos que de una u otra forma tenían contacto directo con la carcasa original, la cual cumplía esta función. 
\end{enumerate}
