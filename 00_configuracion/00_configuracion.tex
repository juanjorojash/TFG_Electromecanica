\usepackage[utf8]{inputenc}
\usepackage[spanish,es-tabla]{babel} %opción es-tabla para que diga Tabla en vez de Cuadro
\usepackage{float} %Paquete para poder definir la posición de una tabla con "H"
\usepackage{booktabs} %Paquete para que las tablas se vean bonitas
\usepackage{tikz} %Paquete para hacer figuras
\usepackage{pgfplots} %Paquete para hacer gráficos
\pgfplotsset{compat=1.16} %Compatibilidad para gráficos
\usepackage{cite}% Paquete para realizar citas
\usepackage{geometry}%Paquete utilizado provisionalmente para dar formato preliminar al documento
\geometry{left=35mm,right=20mm,top=30mm,bottom=30mm,headheight=15pt}
\usepackage{graphicx}
\usepackage{multirow}%Permite realizar configuracion multifila en tablas
\usepackage{gensymb}%Permite agregar el simbolo de grados %use \siunitx para eso
\usepackage{xurl}%Para hipervinculos
\usepackage[T1]{fontenc}
\usepackage[intoc, spanish]{nomencl}
\usepackage{appendix}
\usepackage{subfigure}
\usepackage{amsmath}
\usepackage{pdfpages}
%\usepackage{showlabels}
\usepackage{siunitx}\sisetup{add-arc-second-zero,output-decimal-marker = {,},number-unit-separator=\text{\,}}
\usepackage[colorlinks = true,
            linkcolor = blue,
            urlcolor  = blue,
            citecolor = blue,
            anchorcolor = blue]{hyperref}
\usepackage{textcomp}
%Comandos para darle formato a la hoja
%\geometry{width=15.5cm, marginpar=1cm}
\usepackage{tabularx}
\setlength{\parskip}{1.5mm}
\usepackage{lipsum}